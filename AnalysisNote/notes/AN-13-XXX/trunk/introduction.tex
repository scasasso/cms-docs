%%%%%%%%%%%%%%%%%%%%%%%%
\section{Introduction}
%%%%%%%%%%%%%%%%%%%%%%%%
%% \FIXME Outline of Introduction
%% \begin{itemize}
%% \item bla-bla on uncertainties in track reconstruction
%% \item event/track selection
%% \item outline of the structure of the note
%% \end{itemize}
%The measurement of track momentum is affected by the reconstruction \ldots
%capability and our limited knowledge about the physical configuration
%of the detector. The main sources of biases in the momentum measurement are from residual misalignments and
%weak-modes, imprecisions in the magnetic field model, mismodelings of the material
%distribution/density. In this note the MuScleFit algorithm~\cite{XXX} is used to correct thise effects and extract an estimate %of the transverse momentum resolution. The method is based on an unbinned maximum
%likelihood fit to calibrate the data with a reference model of the J/y.
%This note is organized as follows: first a 
A precise and unbiased measurement of charged track parameters, most important among these
the transverse momentum $p_{T}$, is crucial for the CMS experiment, which is expected to deliver
important precision measurements such as the top quark and $W$ boson masses, or the mass of
the newly discovered Higgs boson. \\
The transverse momentum determination is affected by the details of the detector integration and
operation, in particular the alignment of the silicon tracker detector and the muon chambers, 
the material budget in the sensible volume and the knowledge of the magnetic field. 
Given these sources of bias, the calibration of charged track momentum as well as the
investigation of the residual systematic effects are mandatory for all the CMS analysis aiming
at a precision measurement, such as the ones mentioned above.\\
The MuScleFit algorithm~\cite{CMS_AN_2010-059} implements a multi-dimensional likelihood fit to 
the spectrum of dimuons from the decay of
reference resonances, such as $J/\psi,\, \Upsilon(1S),\, Z$, to determine biases in the momentum assigned to
charged tracks and to estimate its resolution. More details on the likelihood construction and 
the technical implementation in the CMS software can be found elsewhere~\cite{CMS_AN_2010-059}.\\
This note describes the details of the muon momentum scale corrections derived with the MuScleFit 
algorithm using approximatevely $5\,\mathrm{fb}^{-1}$ and $19.5\,\mathrm{fb}^{-1}$ of pp 
collisions data coming from the Run 1 of the LHC at a center of mass energy 
of 7 TeV and 8 TeV respectively and it is organized as follows. Section \ref{sec:calibration} describes 
in details the calibration strategy and the datasets used. Section \ref{sec:validation} presents the 
results of a the validation of calibrated muons. Section \ref{sec:systematics} describes the method and 
results of the estimation of the residual systematic uncertainty on the muon momentum scale 
after the calibration.


%The sample of events Table~\ref{tab:datasets} shows the data

%/MuOnia/Run2012A-13Jul2012-v1/AOD
%/MuOnia/Run2012B-13Jul2012-v1/
%MuOnia/Run2012C-PromptReco-v2/AOD
%Upsilon1SToMuMu_2MuPtEtaFilter_tuneD6T_8TeV-pythia6-evtgen/Summer12_DR53X-PU_S10_START53_V7A-v1/AODSIM
%JPsiToMuMu_2MuPtEtaFilter_tuneD6T_8TeV-pythia6-evtgen/Summer12_DR53X-PU_S10_START53_V7A-v2/AODSI
